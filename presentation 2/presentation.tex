\documentclass[10pt]{beamer}
\usetheme{Madrid}
\usecolortheme{default}
\setbeamertemplate{navigation symbols}{}

% Packages
\usepackage[utf8]{inputenc}
\usepackage{graphicx}
\usepackage{amsmath}
\usepackage{amsfonts}
\usepackage{amssymb}
\usepackage{xcolor}
\usepackage{tikz}
\usepackage{pgfplots}
\pgfplotsset{compat=1.18}
\usetikzlibrary{arrows,shapes,positioning}

% Title page information
\title[Robot Localization with ROS]{\texorpdfstring{Robot Localization with ROS: \\Final Presentation (Steps 1–3)}{Robot Localization with ROS: Final Presentation (Steps 1–3)}}
\subtitle{Mini-Project 1 Final Report}
\author{Nicholas Birch de la Calle (IST1116701) \\
Antonio Maria Trigueiros de Arag\~{a}o Moura Coutinho (IST196837) \\
Gabriel Badan (IST1116537) \\
Jana\'{\i}na da Silva Pacheco (IST1117233)}
\institute{Instituto Superior Técnico}
\date{September 24, 2025}

% Title graphic (IST/Técnico logo) — tries common filenames; skips if not found
\IfFileExists{Tecnico Lisbon Logo.jpg}{\titlegraphic{\includegraphics[height=1.2cm]{Tecnico Lisbon Logo.jpg}}}{%%
\IfFileExists{tecnico.jpg}{\titlegraphic{\includegraphics[height=1.2cm]{tecnico.jpg}}}{%%
\IfFileExists{logo.jpg}{\titlegraphic{\includegraphics[height=1.2cm]{logo.jpg}}}{}}}

\begin{document}

% Title slide
\begin{frame}
\titlepage
\end{frame}

% Outline
\begin{frame}{Outline}
\tableofcontents
\end{frame}

% Team & scope
\section{Team \& Scope}

\begin{frame}{Scope for This Week}
\begin{itemize}
    \item \textbf{Step 1 (Done):} Compare \texttt{/odom} and \texttt{/odometry/filtered} against mocap ground truth
    \item \textbf{Step 2 (In progress):} Built a map from the real robot; created a playback \texttt{.bag}
    \item \textbf{Step 3 (Planned):} Navigation on the built map (AMCL + move\_base) with simple waypoint demo
    \item Same workflow: short syncs, shared checklist, pair-debug on TF/time alignment
\end{itemize}
\end{frame}

% Step 1: GT comparison
\section{Step 1: Ground Truth Comparison}

\begin{frame}{What We Compared (Method)}
\begin{itemize}
    \item Topics: \texttt{/odom} (wheel odom), \texttt{/odometry/filtered} (EKF), mocap ground truth in TF
    \item Aligned frames: \texttt{odom} as fixed frame; mocap link chained via provided TF to \texttt{base\_footprint}
    \item Time-sync: used rosbag timestamps; checked for TF extrapolation and IMU delays
    \item Metrics: lateral/longitudinal error and heading error over time; simple RMSE summary
\end{itemize}
\end{frame}

\begin{frame}{Results (Simple Takeaways)}
\begin{itemize}
    \item \textbf{Tracking:} EKF \texttt{/odometry/filtered} follows mocap closely; reduces wheel-odom drift
    \item \textbf{Turns:} Largest error spikes during fast yaw rotations (IMU bias not fully compensated)
    \item \textbf{Numbers:} Example RMSE (x,y): \textit{[fill in]} m; heading RMSE: \textit{[fill in]} deg
    \item \textbf{Bottom line:} EKF improves consistency and stability vs raw odometry
\end{itemize}
\end{frame}

\begin{frame}{Trajectories vs Ground Truth}
\begin{center}
\IfFileExists{gt_compare.png}{\includegraphics[width=0.9\textwidth]{gt_compare.png}}{\textit{[Insert trajectory plot: mocap (black), odom (orange), filtered (red)]}}

\vspace{2mm}
{\scriptsize Fixed Frame: \texttt{odom}; Data source: recorded rosbag from the TurtleBot3}}
\end{center}
\end{frame}

% Step 2: Mapping
\section{Step 2: Mapping (SLAM)}

\begin{frame}{Mapping Pipeline}
\begin{itemize}
    \item Inputs: \texttt{/scan}, \texttt{/tf}, \texttt{/odom} (filtered odom optional for stability)
    \item Tool: \texttt{gmapping} (2D occupancy grid)
    \item Procedure: teleop + slow loops; then saved map and a playback bag containing map + robot motion
    \item Validation: visualized map in RViz; checked TF continuity during playback
\end{itemize}
\end{frame}

\begin{frame}{Resulting Map (Preview)}
\begin{center}
\IfFileExists{map.png}{\includegraphics[width=0.88\textwidth]{map.png}}{\textit{[Insert occupancy grid screenshot with robot trajectory]}}

\vspace{2mm}
{\scriptsize Resolution: \textit{[fill in]} m/px; Area covered: \textit{[fill in]} m$^2$; Loop closures: \textit{[observed/rare]}}
\end{center}
\end{frame}

\begin{frame}{Playback Bag and Demo}
\begin{itemize}
    \item Created a \texttt{.bag} with map and robot playback for reproducible demos
    \item Makes it easy to re-run RViz overlays and evaluate localization on the fixed map
    \item \textbf{Next:} Use this bag to test AMCL and compare pose vs ground truth trajectory
\end{itemize}
\begin{center}
\IfFileExists{bag_playback.png}{\includegraphics[width=0.8\textwidth]{bag_playback.png}}{\textit{[Optional: insert screenshot of playback]}}
\end{center}
\end{frame}

% Step 3 plan
\section{Step 3: Navigation Plan}

\begin{frame}{What We Will Deliver}
\begin{itemize}
    \item \textbf{Localization on Map:} AMCL tuned with correct laser/TF frames
    \item \textbf{Navigation:} \texttt{move\_base} with simple global/local planners; 2–3 waypoint demo
    \item \textbf{Safety:} costmap inflation + obstacle layer; conservative speeds
    \item \textbf{Evaluation:} path tracking error vs planned path; success rate on short routes
\end{itemize}
\end{frame}

% Conclusion
\section{Conclusion}

\begin{frame}{Final Remarks}
\begin{block}{Progress Summary}
Step 1 completed with mocap comparison; Step 2 mapping underway with a usable occupancy grid and playback bag.
\end{block}

\begin{block}{Next Steps}
\begin{enumerate}
    \item Fill in quantitative RMSE numbers and insert figures
    \item Bring up AMCL + move\_base on the saved map
    \item Record short navigation demo and summarize tracking error
\end{enumerate}
\end{block}

\begin{center}
\Large \textbf{Thank you for your attention!} \\
\vspace{0.5cm}
\normalsize Questions?
\end{center}
\end{frame}

\end{document}
